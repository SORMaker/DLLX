\section{项目介绍}
\subsection{现有资料分析}
轴承动力学模型是研究轴承运动和力学特性的重要工具,通过建立轴承的数
学模型来描述其运动和受力情况,可以实现对轴承性能的分析和优化。滚动轴承
仅由四个部件(即内圈、外圈、保持架和滚珠)组成,但在现实中,滚动轴承的
静态和动态行为非常复杂。原因在于不同轴承部件之间的非线性接触以及轴承运
行过程中出现的复杂摩擦机械现象。因此,滚动轴承建模对于获得基本原理的知
识至关重要。在过去的几十年里,许多模型已经可以用于滚动轴承的设计和仿真。
目前,轴承动力学模型的研究已经在国内外取得了非常多的研究成果。下面将从
正常轴承的建模、局部缺陷建模和打滑轴承的建模三个方面进行综述。

首先,就滚动轴承的正常模型而言,国内外研究者已经有大量成熟的研究成
果了。归纳总结,可以将滚动轴承模型分为集总参数模型、准静态模型、准动态模
型、动力学模型和有限元模型[10]。动力学模型会更加接近轴承的运动状况。在动态
模型中,不使用静态约束。所有的过渡和旋转运动用微分方程描述。对于这方面的
研究很早就开始了,瑞典的SKF 公司和美国Franklin 研究所合作拉开了对轴承动
力学的研究,提出了基于套圈控制理论的自由圆环弯曲振动模型[11]。文献[12]提出了滚珠轴承和滚柱轴承最具代表性的动力学模型之一。滚珠轴承中的三种主要
相互作用以及滚柱轴承中的四种主要相互作用。

其次,在正常的轴承研究之上,有学者对滚动轴承局部缺陷特征的动力学模
型进行了大量的研究。文献[8]提出了用半正弦函数和三角函数描述缺陷区
滚动元件运动状态的数学建模方法。文献[9]提出了一种基于矩形脉冲函数表
征轴承缺陷冲击特性的方法。他们在模型中考虑了缺陷位置的影响。文献[10]使用切线和矩形函数来描述缺陷轴承的偏转激励。文献[11]提出了一
种计算带有矩形缺陷的双列轴承准静态载荷分布和变刚度的方法。文献[12]建
立了一种通过计算滚动元件通过矩形缺陷的位移来估计滚动轴承缺陷大小的方法。
文献[13]与[14]将缺陷特征描述为一个锋利的矩形。

最后,还有学者关注轴承动力学分析的分支打滑现象。该部分研究者考虑了
实际情况下的轴承打滑现象,对轴承的打滑行为做了研究分析。轴承打滑是代表轴承动态性能的另一个重要指标,在轴承运行中经常观察到。它很可能对滚动元件和滚道造成划痕和磨损,并严重影响轴承的工作性能和使用寿命,
特别是在剧烈的载荷振荡和快速的速度变化下。因此,随着研究工作的深入和深入,轴承
打滑现象越来越受到人们的重视。
\subsection{失效初步原因诊断}
如图\ref{f1} 所示, 轴承在正常使用一段时间后突然发生失效
断裂,我们项目组对本次失效提出了我们的问题,本次失效是由于疲劳断裂、过载断裂、腐蚀断裂还是材料工艺原因导致的断裂?

\begin{figure}
    \centering
    \includegraphics[width=0.6\linewidth]{figures/C2/1.jpg}
    \caption{轴承失效断裂}
    \label{f1}
\end{figure}
轴承零件中的任何一个都可能出现故障,这些故障通常是单点缺陷,如切屑
或凹痕。当这些元件相互移动时,这些缺陷与轴承中的其他元件周期性接触,并且
在每个接触处,它们可以在整个结构中激发高频共振。滚动轴承损坏可能导致滚
动轴承降低工作效率甚至完全失效。只有当运行和环境条件以及轴承布置的细节
完全一致时,轴承布置才能有效运行。轴承损坏并不总是由轴承本身造成的。由
于轴承材料或工艺缺陷造成的损坏属于例外。

初步讨论,我们组认为本次轴承失效的原因大概由以下几点:

(1) 润滑不足或污染:润滑不足或使用了不合适的润滑剂可能导致滚动轴承的过早失效。此外,当润滑油或脂中含有污染物时,也可能损伤轴承。

(2) 超载:超过轴承设计的额定负载或短时间内的大负荷可能导致轴承的损坏或过早的疲劳失效。

(3) 安装不当:如果轴承在安装时受到损伤、施加不均匀的载荷或存在对齐问题,可能会导致轴承的早期失效。

(4) 环境因素:如高温、湿度、化学物质(如酸、碱)等,可能会对轴承材料产生腐蚀或其他形式的损害。

(5) 外部污染:外部尘埃、金属屑或其他颗粒物可能进入轴承内部,导致轴承的磨损和损坏。

总之,我们初步得出结论,判定本次失效的主要为安装不当导致在安装时轴承受到损伤同时也受到了外部污染等化学因素的影响造成。
\subsection{本项目的短期解决方法}
在我们小组进行初步原因诊断后,我们立刻制定了短期解决措施,将本次失效损失降到最低。我们将从以下方面进行初步解决

(1) 更换轴承: 最直接的方法是更换失效的轴承。确保使用正确规格和品质的轴承来替换。

(2) 润滑处理: 如果发现轴承润滑不足或质量有问题,我们可以及时添加适当的润滑油或脂。确保润滑油或脂的品质和量都符合规定。

(3) 调整安装: 检查轴承的安装状态,确保其正确对中并且预紧力适当。如果需要,进行适当的调整或校正。

(4) 临时支撑: 在更换轴承之前,如果可能的话,可以考虑添加临时支撑或支撑结构,以减少负载并延长设备的使用寿命。

(5) 降低工作条件: 在进行修复之前,可以考虑降低负载、转速或其他工作条件,以减少对轴承的进一步损伤。

需要注意的是,虽然本报告通过采用以上的方法帮助暂时解决轴承断裂的问题并确保设备的正常运行,但本报告将会进一步对轴承的数学模型以及元素进行相应的分析,以避免未来的失效和损坏。
\subsection{本项目的长期解决方法}
本报告旨在利用数据分析和电镜检测相结合的方式,对滚动轴承的问题进行全面分析。近些年,制造行业一直在研究如何利用数学模型、计算机模拟和实验方法来优化轴承的设计。经过深入研究,总结出四种主要的轴承故障检测手段:振动监测、声学监测、温度监控以及磨损颗粒检查。特别是振动监测,其应用范围非常广泛。

在各种监测方法中,振动分析成为了最受欢迎的技术,因为每个轴承缺陷都会导致滚动部件的特定振动,从而产生连续的振动脉冲。这种振动不仅有低频部分,还有与冲击有关的高频组成。除了使用振动数据,许多研究者还研究了决定轴承行为的物理定律,并尝试用数学方程式来描述这些系统。此外,通过电镜技术,可以对轴承的结构、组件和细节进行详细观察和分析,从而更有效地预防轴承未来可能出现的故障和损伤。下面将对基于数据的故障诊断分析方法分类进行着重说明。

\subsubsection{轴承动力学建模}
本报告将首先对多自由度轴承动力学模型进行了详细介绍。随后,通过研究轴承波纹度、接触刚度以及赫兹接触力理论,建立具有两个自由度的轴承的正常动力学方程。此外,对局部故障下的轴承动力学模型进行了深入研究,运用分段函数揭示了球在经过故障部分时的激励过程,由此推导出了缺陷故障轴承的动力学方程。最终,采用四阶龙格库塔数值解法求解这一二阶方程。
\subsubsection{轴承的早期微小故障检测}
目前的文献研究表明,轴承的动力学模型展现出多输入频率和非线性的特性。基于这种非线性特性,我们可以对轴承的运动行为进行深入探究[15]。在这种非线性体系中,确定了确切的输入频率及其数量后,可以进一步确定主要的输出频率。尽管存在众多输出频率,但与输入频率密切相关的输出频率往往携带了大部分的能量信息,这些输出频率的特征值相对较高。因此,这些核心输出频率可以代表整体的输出特性。
\subsubsection{输入频率未知的轴承故障检测}
当滚动体存在滑动摩擦时,滚动体的自旋速度无法简单地通过线性关系来确定。因此,本部分主要探讨的是在输入频率不确定的情况下的轴承故障检测。通过信号的采集,我们可以推断输出频率,尽管在已知非线性阶数的前提下,我们可能对基频一无所知,或者在非线性系统结构上并不清楚的情境下进行输入频率的估算和故障检测。该部分的核心挑战在于如何有效地利用互调结构来增强信号的估计能力。

\subsection{小结}
报告的第二部分对轴承现有分析方法、失效的初步原因诊断、本项目的短期解决方法以及本项目的长期解决方法进行了综述,同时针对目前研究内容的不足之处,本部分提出利用基于轴承动力学建模、早期微小故障诊断和频率未知下的轴承故障检测器的数据故障诊断方法和电镜观测的元素与端口分析方法结合的断裂分析方法,以避免轴承未来的失效和损坏。
