\section{背景介绍}

\subsection{项目背景及意义}
随着社会不断地的向前发展,现代工业的生产力急剧上升。与此同时,机械
设备正朝着高速、精密、大型及智能方向快速发展,机械设备呈现出规模大、结
构复杂的形态。自然地,此种形态导致与机械设备工作状态相关的内、外因素数
量增加,映射关系更加复杂多样。若实际工况中,机械设备由于元件故障而产生
级联效应,极易导致生产秩序紊乱,继而造成巨大的经济损失和重大事故发生,
甚至人员死伤[1]。旋转机械作为机械设备中举足轻重的构成部件,被广泛应用于
制造、医疗及能源等各个行业,故其状态监测与故障诊断是机械领域的重点。在
诸多旋转机械中,滚动轴承作为不可或缺的构件,被视为用于动态元件和固定元
件连接的“关节”。此外,因其装配方便、效率高、摩擦阻力小、润滑容易实现
等优点,在旋转机械中广泛应用[2-3]。

作为实际工况中使用频率最高的必不可少的设备,滚动轴承也成为发生故障
最频繁的元件之一。因此,其运转状态通常直接关系到整个设备的精度、可靠性
及寿命[4-5]。但在实际工程中,滚动轴承在机械设备中起承受载荷和传递载荷作用,
工作条件最为恶劣,由于其工作面长期受到接触应力的反复作用,极大概率会诱
发磨损、浸蚀及裂纹等失效形式[6]。若已诱发的失效未能及时得到维修,则会进
一步发展导致轴承断裂彻底丧失功能,从而引发灾难性事件。据有关资料统计,
由滚动轴承失效导致的电机故障率约为44\%,约有20\%齿轮箱故障是轴承故障诱
发。此外,机械故障中,70%左右源于振动故障。在振动故障中,约30%故障的
诱因是轴承失效[7]。

以此可推断,滚动轴承安全平稳运转是保证机械设备乃至整
个生产工况有条不紊运行的必要前提和关键因素。为此,针对滚动轴承工作状态
的有效监测和诊断手段需不断探索,以便更准确把握其故障产生和演变机理,旨
在及时挖掘潜藏故障,规避机械设备重大故障的发生,进而避免人力和物力的巨
大损失,具有显著的经济效益和安全效益。此外,滚动轴承由于其工作条件恶劣
不但是易损零件,而且寿命具有很大的离散性,故滚动轴承与许多旋转机械中的
故障相关联。总结来说,探索和研究滚动轴承的状态监测与诊断方法在理论研究和实际运用层面都极具意义。
虽然轴承故障诊断技术日趋成熟,但绝大多数方法只能诊断出重大故障。遗
憾的是,在实际过程中,当轴承的运行状态发生异常时,如存在微点蚀、微裂纹
等微小故障时,轴承振动加速度信号时常存在幅值低、易被未知扰动和噪声掩盖
而导致故障特征信号不明显的特点,给轴承运行状态的安全监控造成了极大困难。
然而,轴承显著性故障都起源于微小故障,若可及时诊断轴承微小故障,可有效
预防重大事故发生,减少巨大的轴承损坏造成的损失。因此,针对轴承故障特征
不明显的微小故障采取相应方法进行故障特征提取及诊断,以保证轴承故障诊断
及安全监控的精确性,受到国内外学者的极大重视[8-9]。
\par
%轴承的缺陷影响不是一概而论的,轻则是产生剧烈振动和噪音,导致生产受影响;重则导致严重的事故,从而带来重大的经济损失和人员伤亡。在现实生活中有多例惨烈的事件。
%在航空方面,小小的轴承故障也会带来严重的整机坠毁事故:
%在2000 年,法国空中客车A330 飞行员试图对自动驾驶系统进行调整,但由于机
%身姿态发生异常,导致飞机坠毁。事故的原因之一是由于飞机发动机轴承故障,导
%致机身失去平衡;在电力设备事故方面,轴承故障可能会导致发电机等设备的运
%转不稳定或停机,进而导致电力系统的失效或事故:2013 年印度发生的一起大规
%模停电事故就是由于几个州的电网因为轴承故障而发生故障,导致全国范围内的
%停电。在火车事故方面,轴承故障可能会导致车轮脱轨,进而导致火车失控或撞击
%其他火车或物体:2017 年加拿大魁北克省的一次火车事故就是由于轴承故障导致
%车轮脱落,使火车失控,最终导致多人死亡和受伤。除此之外,因为轴承的使用范
%围广泛,轴承故障往往还会导致建筑设备事故、医疗器械事故和船舶事故等。

%当轴承的运行状态发生轻微异常
%时,如存在微点蚀、细小裂纹、微磨损等微小故障时,轴承振动加速度信号时常
%存在幅值冲击弱、故障特征隐蔽性较强的特点。更值得注意的是,在实际工况中,
%故障振动信号产生过程通常都伴随着较强的背景噪声,极易掩盖微小故障特征,
%给轴承运行状态的安全监控造成了极大困难。因此,对故障特征信号不明显的含
%噪微小故障信号序列构建合理的消噪模型、故障特征提取机制及诊断算法,可有
%效预防重大故障发生,从而确保轴承健康平稳地运行。故而,针对背景噪声下微
%小故障诊断方法研究引起广大学者的极大关注和讨论。本文选取滚动轴承为研究
%目标,鉴于振动信号是其运行状态最直接的体现,并基于振动信号的处理层面探
%索了轴承微小故障振动信号除噪、特征提取及特征识别三个重要环节
\par
探究轴承故障可能原因并进行轴承失效分析,不仅可以积累发现和诊断轴承故障的经验,以保证机械系统的稳定性、可靠性和寿命,同时也可以降低维护成本,提高工作效率。
因此本报告将以滚动轴承为研究对象,将从宏观及微观角度进行失效分析。
\begin{figure}[h]
\centering
\subfigure[滑动轴承]{
\includegraphics[width=0.3\linewidth]{figures/C1/plain_bearing.jpg}
}
\quad
\subfigure[滚动轴承]{
\includegraphics[width=0.3\linewidth]{figures/C1/bearing1.jpg}
}
\caption{不同类型的轴承}
\label{p1}
\end{figure}

\subsection{轴承分类及工作原理}
滚动轴承微小故障诊断仍旧面临诸多的困难和阻碍,需在已有框架基础上进
一步构建更精确的诊断模型。为此,对滚动轴承的故障演变机理展开研究是对微
小故障进行精细化诊断的必要前提。因此,本报告首先就滚动轴承基本结构以及失效形式等特点展开讨论。

一般来说,轴承基本上可以分为两种类型,如图~\ref{p1}所示,包含滑动轴承和滚动轴承。滑动轴承依靠滑动摩擦而工作。滑动轴承包括直线轴承和滑动轴承。滚珠轴承和滚柱轴承统称为滚动轴承,是常用的机械元件。滚动轴承是一种机械部件,允许轴在各种应用中旋转。滚动轴承可以在低摩擦条件下运行,并且适用于要求卓越耐久性的高速轴速度。滚动轴承由不同的部分组成:一个外圈、一个内圈、在重动载荷和相对高速下接触的滚动元件,以及可选的围绕这些滚动元件的保持架。
%%%%%%%%%%%%%%

滚动轴承在实际工况中起到关节的重要作用,是现在生产设备必不可少的部
件之一。滚动轴承的基本架构主要包括内圈、外圈、滚动体和保持架四个部件,
其整体结构和详细结构见图~\ref{p3}。当滚动轴承被运用到具体生产时,通常安装外圈
于轴承底座、箱体等物体上,运行时处于静态或者相对静态的状态,而内圈连接
到旋转轴的轴颈并随之转动。此外,为降低接触受力和轴向位移,在内圈和外圈
上设计了凹槽滚道。内、外圈的凹槽滚道是滚动体的滑动轨道,通常设计滚动体
形状为圆形或者柱形,以此降低运动摩擦。滚动体是用来衔接动态部件和静态部
件的关键部件,其尺寸大小、外形和个数对整个滚动轴承的负荷水平起到关键性
作用。保持架被用以保证滚动体等间距分布和均衡受力,因而很好地预防了滚动
体发生脱落或者碰撞、磨损的现象。

%运动物体与支承物之间的接触点在不断地变化的摩擦为滚动摩擦。滚动轴承正是基于滚动摩擦工作。滚动轴承的具体结构如图~\ref{p3}所示,滚动轴承一般由外圈、内圈、滚动体、保持架组成。
%滚动体通常通过轴承内的保持架在两个套圈之间均匀布置,其形状、尺寸和数量会影响轴承的承载能力和性能。除了使滚动元件均匀分离外,保持架还可以起引导滚动元件旋转和改善轴承内部润滑性能的作用。
%滚动轴承是借助在内、外圈之间的滚动体滚动实现传力与滚动的。内圈紧密于轴径上,外圈与轴箱之间允许有少许的转动,当车轮转动时内圈随轴径转动,同时带动滚动体转动,滚动体一方面沿内外圈轨道作公转,另一方面绕自身轴心作自转。它们之间的接触点是在不断地变化的,如图~\ref{p4}所示,a, b作相反方向的转动。b代表着滚动轴承中的滚动体。
%\begin{figure}
%\centering
%\subfigure[自行车中的轴承]{
%\includegraphics[width=0.45\linewidth]{figures/C1/bike_zhou_cheng.jpg}
%}
%\quad
%\subfigure[燃气轮机中的轴承]{
%\includegraphics[width=0.45\linewidth]{figures/C1/qi_lun_ji.jpg}
%}
%\quad
%\subfigure[溜冰鞋中的轴承]{
%\includegraphics[width=0.45\linewidth]{figures/C1/hanbingxie.png}
%}
%quad
%\subfigure[变速箱中的轴承]{
%\includegraphics[width=0.45\linewidth]{figures/C1/bian_su_xiang.jpg}
%}
%\caption{各种应用中的滚动轴承}
%\label{p2}
%\end{figure}

\begin{figure}
    \centering
    \includegraphics[width=0.6\linewidth]{figures/C1/bearing3.jpeg}
    \caption{滚动轴承结构示意图}
    \label{p3}
\end{figure}
%\begin{figure}
%  \centering
%   \includegraphics[width=0.6\linewidth]{figures/C1/shi_yi_tu.jpg}
 %   \caption{滚动摩擦原理}
  %  \label{p4}
%\end{figure}


\subsection{轴承的主要失效形式}
滚动轴承是生产过程中最常使用的部件之一,但在实际长期运行过程中,由
于载荷下各部件的相互摩擦、恶劣的工况环境及安装工艺欠佳等缺陷,不可避免
地出现故障。随着其不断运行,故障程度不断的积累,最终将会发生重大故障,
直接影响轴承运行状态,继而令轴承完全失效,造成设备严重损坏。如图~\ref{p5}所示,滚动轴承的
故障主要包括以下几种形式。

(1)磨损失效

磨损失效是滚动轴承故障形式中出现频率最高的一种故障形态。在滚动轴承
的工作过程中,滚动体随着其运转不断地在内、外圈间的滑槽滑动,长期的滑动
导致部件表面出现磨损。此外,在滚动轴承负荷运行时,若金属粉末或者其它坚
硬杂物误入轴承内部,将会对轴承工作表面造成一定程度的磨损。

(2)疲劳失效

疲劳失效时常发生于轴承的表面。由于滚动体表面和滚道长期承受着重物和
相互挤压的压力,导致在最大剪应力处形成裂纹。裂纹是疲劳失效最初期的故障
体现,裂纹进一步发展则成为剥落坑,最终导致剥落从而引起轴承失效,这种失
效即疲劳失效。

(3)腐蚀失效

腐蚀失效是由金属部件松动、缝隙夹渣和表面缺陷等诱发的微电池作用而触
发。点腐蚀是腐蚀的初期症状,随着腐蚀面的不断扩大,最终造成表面剥落使轴
承失效,即腐蚀失效。

(4)塑性变形失效

轴承在运行中,若遭受较大的冲击负荷或者坚硬颗粒杂物的侵入,将会导致
轴承部件表面的凹痕、划痕或压痕的产生,称之为塑性变形。随着这种变形的不
断增大,最终将引起剥落导致轴承失效,即塑性变形失效。

(5)胶合失效

由于高转速、重负荷、润滑程度不足,轴承会因为摩擦产生巨大而短暂的温
度,导致部件表面烧损。在这种巨大的温度下,滚动体和滚道的局部表面粘连到
一起导致轴承失效,即胶合失效。

(6)断裂失效

在长时间运转过程中,滚动轴承会因为承力过大、转速太大、润滑程度不够、
加工误差或者安装不当产生超过其承受上限的热应力。正是因为这种过大的热应
力造成了轴承元件的断裂。

%%%%%%%
\begin{figure}[htbp]
\centering
\subfigure[轴承磨损]{
\includegraphics[width=0.45\linewidth]{figures/C1/mo_sun.jpeg}
}
\quad
\subfigure[轴承裂纹]{
\includegraphics[width=0.45\linewidth]{figures/C1/lie_wen.png}
}
\quad
\subfigure[轴承腐蚀]{
\includegraphics[width=0.45\linewidth]{figures/C1/fu_shi.png}
}
\quad
\subfigure[轴承碎裂]{
\includegraphics[width=0.45\linewidth]{figures/C1/sui_lie.jpeg}
}
\caption{轴承的主要失效形式}
\label{p5}
\end{figure}
%%%%%%%%%
% \begin{figure}[h]
% \centering
% \subfigure[轴承磨损]{
% \includegraphics[width=0.3\linewidth]{figures/C1/mo_sun.jpeg}
% }
% \quad
% \subfigure[轴承裂纹]{
% \includegraphics[width=0.3\linewidth]{figures/C1/lie_wen.png}
% \quad
% \subfigure[轴承磨损]{
% \includegraphics[width=0.3\linewidth]{figures/C1/mo_sun.jpeg}
% }
% \quad
% \subfigure[轴承裂纹]{
% \includegraphics[width=0.3\linewidth]{figures/C1/lie_wen.png}
% }
% \caption{轴承失效形式}
% \label{p1}
% \end{figure}